% !TeX program = lualatex

% CLASS OPTIONS (all optionals, order is irrelevant)
%   - pdfa: specifies if the pdf should be of `pdfa` type
%   - banner={banner1,banner2,...}: specifies the banners to use (order of the banner is important)
%   - handout: removes the \pause
%   - nolight: specifies if you dont want a light font (default)
\documentclass[banner={aau},handout]{presenta}

\newcommand{\variables}[2]{#1_1, \ldots, #1_{#2}}
\newcommand{\ring}{\mathbb{K}[\variables{x}{n}]}
\newcommand{\I}{\mathcal{I}}
\newcommand{\ideal}{\langle \lt{g_1}, \ldots, \lt{g_t} \rangle}
\newcommand{\lc}[1]{\mathsf{LC}(#1)}
\newcommand{\lm}[1]{\mathsf{LM}(#1)}
\newcommand{\lt}[1]{\mathsf{LT}(#1)}
\newcommand{\qeq}{\stackrel{?}{=}}

\newcommand{\field}{\mathbb{F}}
\newcommand{\fp}{\mathbb{F}_p}
\newcommand{\fw}{\mathbb{F}_{2^\omega}}
\newcommand{\perm}{\mathcal{P}}


\title{Some simple title}
\author{me}

\subtitle{The subtitle}
\supervisor{Prof A}
\cosupervisor{Prof B, Prof C}
\tutor{Prof D}

\begin{document}

\maketitle

\tableofcontents

\section{Section}

\begin{frame}{A long frame title with subtitle}{a long subtitle}
  Here's some list
  \begin{itemize}
    \item first;
    \pause
    \item second;
    \pause
    \item third.
  \end{itemize}
\end{frame}

\begin{frame}[right]{short}{sht}
  A \textit{right sided} frame title with \textbf{enumerate}:
  \begin{enumerate}
    \item item 1;
    \item second item;
    \item last item.
  \end{enumerate}
\end{frame}

\begin{frame}{Figure example}
  Here we import a figure:
  \begin{figure}[H]
    \begin{tikzpicture}

  \draw[rounded corners,thick,accentmedium,fill=accentlight!40] (0,0.4) rectangle (2,-0.4) node[midway,black] {A blocked text};

\end{tikzpicture}

    \caption{An example of \texttt{tikz}.}
  \end{figure}
\end{frame}

\begin{frame}
  \begin{columns}
    \hfill
    \begin{column}[t]{0.40\paperwidth}
      \justifying
      An example of frame without title, with a split layout and a citation~\cite{example1}.
    \end{column}
    \hfill
    \begin{column}[t]{0.40\paperwidth}
      \justifying
      All this text is justified\footnotemark, and height is adjusted on higher text level.
    \end{column}
    \hfill
  \end{columns}
  \footnotetext{simple as that}
\end{frame}

\section{Another helpful section}

\begin{frame}{Math examples}
  $0x3 + 0^3 + I_{m^\prime} - \rho^{24} = \widehat{16}$

  \begin{equation*}
    f = - 5x^3 + 7x^2z^2 + 4xy^2z + 4z^2  
  \end{equation*}

  $x^2 \notin \langle \lt{f_1}, \lt{f_2} \rangle$

  \begin{definition}[\texttt{Square}]
    The function $S: \field^2_p \rightarrow \field^2_p$ is a squaring operation followed by the addition of a pseudo-random round constant $\gamma$ and is defined as follows:
    \begin{equation*}
      S_i(x_L, x_R) \rightarrow \left( x_R + \frac{(x_L)^2}{\sigma} + \gamma_i, x_L \right),
    \end{equation*}
    % with $\sigma$ a Montgomery constant.
    with $\sigma$ a constant.
  \end{definition}
\end{frame}

\standout{If you wanna make text standout :)}

\references

\end{document}
