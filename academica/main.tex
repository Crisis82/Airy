% CLASS OPTIONS (all optionals, order is irrelevant)
%   - pdfa: specifies if the pdf should be of `pdfa` type
%   - light: specifies if we want to use a lighter font
%   - banner={banner1,banner2,...}: specifies the banners to use (order of the banner is important)
\documentclass[light,banner={aau,uniud}]{academica}
\usepackage{lipsum} % for dummy text in the example

% MANDATORY FIELDS
\title{Title example}
\author{Me} % or
%\candidate{Me, but candidate}

% OPTIONAL TITLE FIELDS (if no argument is provided, it will be skipped)
\overtitle{My course of Studies}
\supervisor{Prof.\ A}
\cosupervisor{Prof.\ B,Prof.\ C}
\tutor{Prof.\ D}
\institutecontacts{\uniudcontacts}
\authorcontacts{
  Me myself\\
  address 1, Random (ZZ)\\
  +012 3456789\\
  \href{mailto:example@email.com}{example@email.com}\\
  \url{example.com}
}

\begin{document}

\frontmatter

\maketitle

\begin{dedication}
	My dedication.
\end{dedication}

\acknowledgements

Example of acknowledgements.

\abstract

This paper has been generated to show the template structure.

\tableofcontents

\mainmatter

\chapter{First chapter}

Example of different text style:
\begin{itemize}
  \item \textbf{Bold text}
  \item \textit{Italic text}
  \item \textsl{Slanted text}
  \item \texttt{Typewrite text}
  \item \textsc{Small Caps text}
  \item $x + x = \sum_{i=1}^{16} x^2 \cdot 2$
\end{itemize}

\lipsum[1-4]

\section{First sec}

\lipsum[5-8]

\subsection{Subsection}

\lipsum[9]

\chapter{Second Chapter Title}

\lipsum[11-14]

\subsubsection{SubSubsection}

\lipsum[15-17]

\begin{definition}[mydef]
  A simple definition.
\end{definition}

\appendix

\chapter{First chap of appendix}

\backmatter

%%\summary
%
%%% Bibliografia (praticamente obbligatoria)
%\bibliographystyle{plain_\languagename}%% Carica l'omonimo file .bst, dove \languagename è la lingua attiva.
%%% Nel caso in cui si usi un file .bib (consigliato)
%\bibliography{academica}
%%% Nel caso di bibliografia manuale, usare l'environment thebibliography.
%
%%% Per l'indice analitico, usare il pacchetto makeidx (o analogo).

\end{document}
